\documentclass{article}
\usepackage{ctex,soul,float,listings,enumerate,hyperref,url,amsfonts,amsmath,graphicx,multirow}
\usepackage{xcolor,tocloft,theorem,numerica,amsmath,mathrsfs}
\usepackage{changes}
\usepackage{fancyhdr}
%%%%%%%%%%%%%%%%%%%%%%%%
\definecolor{AnatationColor}{RGB}{0,139,0}
\lstset{
    backgroundcolor = \color{white},    % 背景色:白色
    basicstyle = \small\ttfamily,           % 基本样式 + 小号字体
    rulesepcolor= \color{white},             % 代码块边框颜色,白色
    breaklines = true,                  % 代码过长则换行
    numbers = left,                     % 行号在左侧显示
    numberstyle = \small,               % 行号字体
    keywordstyle = \color{blue},            % 关键字颜色
    commentstyle =\color{AnatationColor},        % 注释颜色
    stringstyle = \color{red!100},          % 字符串颜色
    frame = shadowbox,                  % 用(带影子效果)方框框住代码块
    showspaces = false,                 % 不显示空格
    columns = fixed,                    % 字间距固定
    %escapeinside={<@}{@>}              % 特殊自定分隔符:<@可以自己加颜色@>
    morekeywords = {as},                % 自加新的关键字(必须前后都是空格)
    deletendkeywords = {compile}        % 删除内定关键字;删除错误标记的关键字用deletekeywords删!
}
\hypersetup{
    colorlinks=true,
    linkcolor=red,
    filecolor=blue,      
    urlcolor=blue,
    citecolor=cyan,
}
\newtheorem{definition}{定义}
\graphicspath{{./Image/}}
%%%%%%%%%%%%%%%%%%%%%%%%
\author{ZeitHaum}
\date{\today}
\title{算法-图上问题}
\pagestyle{fancy}
%%%%%%%%%%%%%%%%%%%%%%%%
\begin{document}
    \pagenumbering{Roman}
    
    \maketitle
    \newpage 
    \tableofcontents
    \newpage
    \setcounter{page}{1}
    \pagenumbering{arabic}
    \section{基本概念}
    统一符号声明:
    \begin{enumerate}
        \item 正权图
        \item 阶数
        \item 相邻、邻域
        \item 握手定理:$\sum_{v}d(v) = 2|E|$
        \item 孤立点、奇点、偶点、叶节点、支配点
        \item 最小度$\delta(G)$、最大度$\Delta(G)$
        \item 出度$d^{+}(v)$, 入度$d^{-}(v)$
        \item $k-$正则图
        \item 自环、重边、简单图、多重图
        \item 路径 
        \item 迹:不重复经过结点
        \item 回路: 首尾相同,环:当且仅当首尾相同
        \item 子图、导出子图(边点对应...)、生成子图(点集相同)
        \item 连通图、连通分量、强/弱连通图、强/弱连通分量
        \item 割(此部分较为复杂...)
        \item 稀疏图、稠密图
        \item 补图(互补),反图(反向)
        \item 完全图、零图
        \item 竞赛图(任意两点间有单向边)
        \item 环图、 菊花图
        \item 轮图(环图+菊花图)
        \item 链,树,基环树/森林
        \item 基环外向树/森林(展开),基环内向树/森林(聚拢)
        \item 仙人掌,沙漠
        \item 二分图:两部分内部无连边,完全二分图
        \item 平面图
        \item 图同构
        \item 交、并、直和
        \item (NP,复杂...)支配集、边支配集、独立集
        \item (复杂...)匹配
        \item (复杂...)点覆盖(NP)、边覆盖
        \item 团、极大团(NP)
    \end{enumerate}

    \section{}

    \section{参考资料}
    [1]. \href{https://oi-wiki.org/graph/}{OI-wiki}
\end{document}
